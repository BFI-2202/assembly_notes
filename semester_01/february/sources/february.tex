\documentclass{article}
\usepackage[utf8]{inputenc}

\usepackage[T2A]{fontenc}
\usepackage[utf8]{inputenc}
\usepackage[russian]{babel}

\usepackage{minted}

\title{Архитектура ЭВМ и язык ассемблера}
\author{Лисид Лаконский}
\date{February 2023}

\begin{document}
\raggedright

\maketitle
\tableofcontents
\pagebreak

\section{Архитектура ЭВМ и язык ассемблера - 01.02.2023}

\textbf{Ассемблер} – \textbf{машинно-ориентированный язык} и позволяет максимально использовать ресурсы процессора.

Ассемблер \textbf{тесно связан с архитектурой}.

Ассемблер – \textbf{мнемоническое отображение команд процессора}.

\hfill

\textbf{Стадии выполнения операции}: \textbf{считывается код команды} — \textbf{считывается первый аргумент} — \textbf{считывается второй аргумент} — \textbf{выполнение операции} — \textbf{пересылка операции}

\hfill

\textbf{Этапы создания программы}:

\begin{enumerate}
    \item Создание текстового файла с мнемонической записью
    \item Создание объектного модуля программы — результат трансляции без привязки к библиотечным модулям и адресам
    \item Компоновка (линковка) — привязка модулей и адресов
    \item Вывод загрузочного модуля
\end{enumerate}

Во время компоновки может быть прикреплен \textbf{отладчик}

Отладчики бывают нескольких видов:

\begin{enumerate}
    \item \textbf{Объектный} — проверяет только сам исполняемый файл
    \item \textbf{Системный} — проверяет работу программы в системе
\end{enumerate}

\hfill

\textbf{BIOS} обеспечивает только ввод и вывод, операционная система — это другое

\textbf{Макросы} подставляются на этапе трансляции

Существует множество трансляторов языка ассемблера: MASM, NASM, TASM и другие

\hfill

\textbf{COM} формат исполняемых файлов не имеет заголовка, его максимальный размер — 64 Кб

\textbf{EXE} улучшен относительно COM, не ограничен по размеру, может содержать несколько сегментов кода

\pagebreak
\subsection{Hello, world}

\subsubsection{TASM/NASM, MS DOS}

\begin{minted}{tasm}
section .text
    org 0x100

    mov ah, 0x9
    mov dx, hello
    int 0x21
    
    mov ax, 0x4c
    mov al, 0
    int 0x21

section .data
	hello DB "Hello, world!", 0xd, 0xa, '$'
\end{minted}

\subsubsection{NASM, UNIX}

\begin{minted}{nasm}
global _start

section .data
    ; String, which is just a collection of bytes, 0xA is newline
    str:     db 'Hello, world!',0xA
    strLen:  equ $-str

section .bss

section .text
_start:
    mov     edx, strLen     ; Arg three: the length of the string
    mov     ecx, str        ; Arg two: the address of the string
    mov     ebx, 1          ; Arg one: file descriptor, in this case stdout
    mov     eax, 4          ; Syscall number, in this case the write(2) syscall: 
    int     0x80            ; Interrupt 0x80        

    mov     ebx, 0          ; Arg one: the status
    mov     eax, 1          ; Syscall number:
    int     0x80
\end{minted}

\end{document}
